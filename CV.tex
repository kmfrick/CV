\documentclass[11pt, a4paper]{article}
\usepackage[T1]{fontenc}
\usepackage[english]{babel}
\usepackage[margin=1cm]{geometry}
\usepackage[utf8]{inputenc}
\usepackage{hyperref}
\hypersetup{colorlinks,breaklinks,
            urlcolor=[rgb]{0.6,0,0},
            linkcolor=[rgb]{0.6,0,0},
            citecolor=[rgb]{0.6,0,0}}
\usepackage{microtype}
\usepackage[skip=3pt]{parskip}
\usepackage[explicit]{titlesec}
\usepackage{fouriernc}
\input{glyphtounicode}
\pdfgentounicode=1
\pagenumbering{gobble}
\newcommand\hreftt[2]{\href{#1}{\texttt{#2}}}
\renewcommand{\familydefault}{\sfdefault}
\usepackage{enumitem}
% List spacing
\setlist{nosep, leftmargin=*}
\setlist{parsep=2pt, topsep=2pt, itemsep=2pt, leftmargin=*}
% Title spacing
\titlespacing\section{0pt}{8pt plus 2pt minus 2pt}{10pt plus 2pt minus 2pt}
\titlespacing\subsection{0pt}{4pt plus 2pt minus 2pt}{8pt plus 2pt minus 2pt}

% Title format
\makeatletter
\def\@maketitle{
  \newpage
  {\LARGE {\textbf{{\@title}} }} \hfill{\footnotesize \href{mailto:kmfrick98@gmail.com}{\texttt{kmfrick98@gmail.com}} |  \url{kmfrick.github.io} | \url{linkedin.com/in/kmfrick/}}
  
  \vspace{-1em}
  \hrulefill
}
\makeatother
\title{Kevin Michael Frick}
\author{}
\titleformat{\section}{\bfseries\Large}{\noindent\thesection}{0pt}{#1~\hrulefill}

\begin{document}


\maketitle

\section*{Education}

\textbf{MSc Economic Theory and Econometrics} \hfill \textbf{Toulouse School of Economics \textbar{} 2022 - 2023}

\textbf{MSc Computer Engineering} \hfill  \textbf{ Bologna University
\textbar{} 2020 - 2022}

\textit{Summa Cum Laude}. Thesis: Autonomous Pricing using Policy Gradient Reinforcement Learning.

Exchanges at \textbf{École Normale Supérieure Paris}, Department of Economics, and at \textbf{UPC
BarcelonaTech}.

\textbf{BSc Computer Engineering} \hfill \textbf{Bologna University
\textbar{} 2017 - 2020}

\textit{Summa Cum Laude}. Thesis: Machine Learning for Semantic Visual SLAM.




\section*{Work experience}


\textbf{Research Assistant \hfill  \hspace{1pt} Paris School of Economics \textbar{} 10/2021 - 07/2022}

\begin{itemize}

\item
  Worked on the design of \textbf{machine learning} models able to
  learn a measure of occupation-to-occupation \textbf{skill distances}
  of job occupational positions from textual and structured data.
\item
  In charge of designing an end-to-end architecture comprising a
  \textbf{language model} and a neural network for task-driven
  \textbf{representation learning} for a \textbf{skill matching} task.
\item

  Project work financed by the French Ministry of Labor, 
  aimed at evaluating re-training programs for workers from distressed occupations to
  \textbf{maximize their re-employment} chances.
  \item
  
  Attended two full-immersion practical \textbf{workshops} organized by the French National Research Center (\textbf{CNRS}) on \textbf{optimizing deep learning models} for high-performance computer clusters.
\end{itemize}

\textbf{Research Intern \hfill \hspace{1pt} University of Twente \textbar{} 03/2020 - 06/2020}
\begin{itemize}

\item
  Researched and deployed deep \textbf{neural networks} for semantic
  segmentation of 3D scenes to improve the localization and mapping
  (SLAM) capabilities of an unmanned aerial vehicle.
\item

  Developed a module for the ROS framework in \textbf{C++ and Python}
  using TensorFlow Lite and ONNX.
\item

  Acquired proficiency in the process of conducting \textbf{independent
  research}, from state-of-the-art studies to the formulation and
  answering of meaningful research questions.
\end{itemize}

\section*{Programming languages and frameworks}
C++, Python, R, MATLAB, Java, C\#, Go, SQL, Stata, PyTorch, TensorFlow, SLURM

\section*{Languages and nationality}

 \textbf{Italian} (native), \textbf{English} (bilingual, IELTS 8.5),
  \textbf{Spanish} (fluent), \textbf{French} (basic). \textbf{Italian} and \textbf{German} dual national.

\section*{Teaching activities}

  Course in \textbf{Argumentation in Artificial Intelligence} \hfill Teaching Assistant, Bologna University, fall 2021


  BSc course in \textbf{Software Engineering}\hfill Teaching Assistant, Bologna University, 2021


  High school course in \textbf{Competitive Programming} \hfill Teacher, Istituto Tecnico Aldini, Bologna, 2020-2021


  High school course in \textbf{Computer Science} \hfill Teacher, Istituto
  Tecnico Aldini, Bologna, 2020-2021


\section*{Scholarships}

TSE Doctoral Track Scholarship for M2 ETE \hfill 2022 - 2023


Collegio Superiore Alumni Scholarship \hfill 2022


Collegio Superiore Fellowship \hfill 2017 - 2022

\section*{Publications}

\textbf{Frick, K. M.} What can economists learn from machine learning? In
\emph{Astrazioni stenografiche. Concetti chiave per vivere consapevolmente la nostra società} (Bononia University Press, 2020). ISBN 9788869236761.


\end{document}

